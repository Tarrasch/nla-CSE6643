\documentclass[a4paper,11pt]{article}
\usepackage[T1]{fontenc}
\usepackage[utf8]{inputenc}
\usepackage{lmodern}
% \usepackage{hyperref} % fucking warnings
\usepackage{graphicx}
\usepackage{rotating}
\usepackage{listings}
\usepackage{color}
\usepackage{listings}
\usepackage{amsthm}
\usepackage{amsmath}
\usepackage{amssymb}
\usepackage{algorithmic}
\usepackage{datatool}

% \newcommand{\encode}[1] {{ {}_{\llcorner}{#1}_{\lrcorner}}}

\title{Numerical Linear Algebra (CSE-6643) - Midterm take home exam}
\author{Arash Rouhani (rarash@gatech.edu) - gtid: 902951864}


\begin{document}

\maketitle

\section{Part I}

blah blah

\section{Part II}

\subsection{a}

Here is an algorithm to generate a matrix $A$ with the given singular values
$\Sigma$. Generate two random unitary matrixes $U$ and $V$ and let $A = U
\Sigma V^*$. The way you generate the unitary matrices could be like this:

Let $P=I$ be a projector which is gonna make sure of the orthogonality of the
vectors we generate. For each $i=1 \to n$, we first randomly create size $n$
row vector $a_i$. Now to ensure that it's orthogonal to all previous vectors,
we multiply it by $P$. To ensure that future vectors will be orthogonal to the
current vector we set $P$ to $P P_{\perp a_i}$. Also, after we have projected
$a_i$ we also normalize it. Hence each vector we create is orthogonal to all
previous ones and it's also of length 1 because it's normalized. Constructing
a matrix consisting of these $n$ vectors will be unitary.

In practice this algorithm is superflous, because this is exactly
how the stable gram schmidt generates $Q$, only that the $a_i$ vectors isn't
random but based on the input, so we create our unitary matrixes by feeding
random data to stable gram schmidt and only looking at it's $Q$ output.

\subsection{b}

I found it interesting to look at these values in the two algorithms $QR$
factorization.

\begin{itemize}
  \item Look at the norms of the produced $Q$s. Ideally they should be $1$.
  \item Define two matrices $\Delta Q$ and $\Delta R$ to $Q_{classic} -
    Q_{stable}$ and $R_{classic} - R_{stable}$ respectivly. Then plot both
    their row and column medians.

\end{itemize}

Let's first look at the data for $n=100$ and then see if
the pattern is partiuclarly different for $n=150$.

\subsubsection{$n=100$}

\begin{tabular}{r|c|c|c|}
  \multicolumn{1}{r}{}
   & \multicolumn{1}{c}{$Q_{classic}$ }
   & \multicolumn{1}{c}{$Q_{stable}$}  \\
   & \multicolumn{1}{c}{$Q_{householder}$} \\
  \cline{2-4}
  $n=100$ & \input{data/norm-q-classi-100.dat}
          & \input{data/norm-q-stable-100.dat}
          & \input{data/norm-q-househ-100.dat}
          \\ \cline{2-4}
  % $n=150$ & \input{data/norm-q-classy-150.dat}
  %         & \input{data/norm-q-stable-150.dat}
  %         & \input{data/norm-q-househ-150.dat}
  %         \\ \cline{2-4}
\end{tabular}


% The 2-norm for is and
% \input{data/norm-q-stable.dat} for .

% The plots 

\subsubsection{$n=150$}

% Now, loop through  $a_i$ $P:= $

% \begin{algorithm}
%   \FOR{$i = 1 \to 10$}
%     blah
%   \ENDFOR
%   \caption{How go generate unitary matrix}
% \end{algorithm}

\end{document}
